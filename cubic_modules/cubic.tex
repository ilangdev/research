% Created 2022-11-25 Fri 20:15
% Intended LaTeX compiler: pdflatex
\documentclass[a4paper, 12pt]{article}
\usepackage[utf8]{inputenc}
\usepackage[T1]{fontenc}
\usepackage{graphicx}
\usepackage{grffile}
\usepackage{longtable}
\usepackage{wrapfig}
\usepackage{rotating}
\usepackage[normalem]{ulem}
\usepackage{amsmath}
\usepackage{textcomp}
\usepackage{amssymb}
\usepackage{capt-of}
\usepackage{hyperref}
\usepackage{ stmaryrd }
\usepackage{ textcomp }
\usepackage{ dsfont }
\author{Ilya Gruzdev}
\date{}
\title{Homotopy properties of cubical $\Sigma$-modules}
\hypersetup{
 pdfauthor={ilyha},
 pdftitle={sample},
 pdfkeywords={},
 pdfsubject={},
 pdfcreator={Emacs 26.3 (Org mode 9.1.9)}, 
 pdflang={English}}

%%% Колонтитулы
\usepackage{fancyhdr} %загрузим пакет
\pagestyle{fancy} %применим колонтитул
\fancyhead{} %очистим хидер на всякий случай
\fancyhead[LE,RO]{\thepage} %номер страницы слева сверху на четных и справа на нечетных
\fancyhead[CO]{R}
\fancyhead[LO]{l} 
\fancyhead[CE]{l} 
\fancyfoot{} %футер будет пустой

\setcounter{tocdepth}{2}

%%% new comannds
\newcommand{\osl}{\oslash}
\newcommand{\rosl}{\reflectbox{$\osl$}}
\newcommand{\mc}{\mathcal}
\newcommand{\rarr}{\shortrightarrow}
\newcommand{\xrarr}{\xrightarrow}

%%% operators
\DeclareMathOperator*{\oid}{\textbf{1}}
\DeclareMathOperator*{\Hom}{\text{Hom}}
\DeclareMathOperator*{\Ho}{\text{Ho}}

%%% commutative diagrams
\usepackage{tikz-cd}
\tikzcdset{row sep/normal=1.5cm}
\tikzcdset{column sep/normal=2.5cm}










\begin{document}

\maketitle

\begin{abstract}
	The main aim of this paper is to study cubical $\Sigma$-modules and their homotopy properties.
\end{abstract}

\tableofcontents

\newpage
\section{$\Sigma$-modules.}
\label{sec:orgb22ab46}

	\subsection{Basic categorical definitions.}
    For convenience, we recall some category theory definitions.
		\subsubsection{Monoidal categories.}
		A symmetric monoidal category is category equiepd a bifunctor $\otimes : \mathcal{C} \times \mathcal{C} \shortrightarrow \mathcal{C}$ called monoid (tensor) product and unit object $\oid$ $\in$ $\mc{C}$
		with some natural transforamtions $ X \otimes Y  \cong Y \otimes X $, 
		$ X \otimes (Y \otimes Z) \cong (X \otimes Y) \otimes Z $, 
		$ \oid \otimes X \cong X \cong \oid \otimes X $.
		These transformations must statisfy an additional coherence conditions [ML98].
		\subsubsection{Category with external $\otimes$-action.}
		Let $\otimes$-category $\mathcal{C}$ is fixed. We can define an external (left) $\otimes$-action 
		($\rosl$) on some category $\mathcal{D}$. This means that we have a bifunctor 
		$ \rosl : \mathcal{A} \times \mathcal{B} \to \mathcal{D} $ togehther some family of functorial
		isomorphisms $ (X \rosl Y) \otimes Z \cong X \rosl (Y \rosl Z) $ statisfied a coherence conditions.
		A right $\osl$-action is defined similarly.
		\subsubsection{Morphisms of $\otimes$-categories.}
		There is a notion of $\otimes$-functor $ F : \mc{C} \rarr \mc{C'} $ between two $\otimes$-categories.
		By defenition, this is functor $F$ with natural ispomorphisms $ F(X \otimes_\mc{C} Y)  \cong F(X) \otimes_\mc{C'} F(Y) $, 
		$ F(\oid_{\mc{C}}) = \oid_\mc{C'} $ compatible with and associtivity and units. \newline
		For two categories $ \mc{D} $, $ \mc{D'} $ with external $ \rosl $-actions $ \mc{C} $, $ \mc{C'} $, respectivly, there is
		a notion of $\rosl$-functors $ G: \mc{D} \rarr \mc{D'} $ compatible with $F$ that is 
		$ G(X \rosl_\mc{D} Y) \cong F(X) \rosl_\mc{D'} G(Y) $ also cmopatible with associativity and units on
		$\mc{D}$ and $\mc{D'}$.
		\subsubsection{Algebras.} An algebra (or a monoid) in $\otimes$-category $\mc{C}$ is a triple $ A = (A, \mu, \epsilon) $
		subject to ususal axiom $ \mu \circ (\oid_{\mc{C}} \otimes \mu) = \mu \circ (\mu \otimes \oid_{\mc{C}}) : 
		\mc{C} \times \mc{C} \times  \mc{C} \rarr \mc{C} $ and $ \mu \circ (\oid_{\mc{C}} \otimes \epsilon) = \oid_{C} =
		\mu \circ (\epsilon \otimes \oid_{\mc{C}}) $. Them is said multiplication and unit, respectivly. An algebra homorphism
		$ f : (A, \mu_{A}, \epsilon_{A}) \rarr (B, \mu_{B}, \epsilon_{B}) $ is a morphism $ f : A \rarr B $ such that
		$ \mu_{B} \circ (f \otimes f) = f \circ \mu_{A} $. Therefore we can define a category algebras in $\mc{C}$,
		denoted by $ Alg(\mc{C}) $.
		\subsubsection{Modules.} Let given an external (left) $\otimes$-action $\mc{C}$ on $\mc{D}$ and algebra 
		$ A = (A, \mu, \epsilon) $ in $\mc{C}$. Then (left) $A$-module in $D$ is by definition a pair $ M = (M, \alpha) $ such that
		$ M \in Ob(D)$, $\alpha : A \rosl M \rarr M $ and there is a conditions 
		$ \alpha \circ (\mu \rosl \oid_M) = \alpha \circ (\oid_\mc{C} \rosl \alpha)$, 
		$ \alpha \circ (\epsilon \rosl \oid_{M}) = \oid_{M} $. A morphism $ f : (M, \alpha_{M}) \rarr (N, \alpha_{N}) $
		between two $A$-modules is morphism $ f : M \rarr N $ in $D$ sompatible with $A$-actions, i. e.
		$ f \circ \alpha_{M} = \alpha_{N} \circ (\oid_{A} \rosl f)  $ . Therefore, $A$-modules in $\mc{D}$ define a category,
		denoted by $ \mc{D}^A $.
		\subsubsection{Monads.} Let's consider a category $Endof(\mc{C})$ of endofunctors of category $\mc{C}$.
		There is a natural monoidal strauture on this category $ F \otimes G = F \circ G $, i.e. composition of functors.
		A monad $\Sigma$ is algebra on this category. A morphism two monads $ \phi : \Sigma \rarr \Sigma' $ is a morphism
		corresponding algebras. A monads over $\mc{C}$ define a category, denoted by $ Monads(\mc{C}) $. Therefore
		$ Monads(\mc{C}) = Alg(Endof(\mc{C})) $.
		
	\subsection{Algebraic monads and generlized rings.}
	%Let's fix a base category $ \mc{C} $ that will be equal to $ Sets $. All of the endfunctors and monads will be conisder 
	%over this category.
		\subsubsection{Algebraic endofunctors and monads.} An endofucntor $ \Sigma : Sets \rarr Sets $ is
	    \textit{algebraic} if it commutes with fltered inductive limits. An algebraic endofuntors is full $ \otimes $-subcategory of
	    category $ \mc{A} = Endof(Sets) $, dnoted by $ Endof_{alg}(Sets) $ or $ \mc{A}_{alg} $. An \textit{algebraic monad} is an
	    algebra (or a monoid) in this category. Let's denote category standart finite sets $ \textbf{n} = \{1,2,...n\} $ as 
	    $\underline{\mathds{N}}$. This is a full subcateofry of $Sets$. Because any set is filtered inductive limits of all its finite subsets and any finite 
	    sunbset isomorphic to some standart set, we have an equivalence between $\mc{A}_{alg}$ and 
	    $ Funct(\underline{\mathds{N}}, Sets) = Sets^{\underline{\mathds{N}}} $. We define an \textit{algebraic moand} as monad in category $ \mc{A}_{alg} $.
	    \subsubsection{Algebraic operations.} For given algebraic monad $\Sigma$ and set $X$ a morphism $ \mu : \Sigma(X) \rarr X$
	    is equivalent to a family of maps $ \{ \mu^{(n)} : \Sigma(n) \times X^{n} \rarr Y \}_{n \geqslant 0} $ subject to conditions $ \alpha^{(m)} \circ (id_{\Sigma(m)} \times X^{\phi}) = \alpha^{(n)} \circ (\Sigma(\phi) \times id_{X^n{}}) : \Sigma(m) \times
	    X^{n} \rarr Y $ for all $ \phi : \textbf{m} \rarr \textbf{n} $, where $ X^{\phi} = Hom(\phi,id_{X}) : Hom(n,X) = X^{n}
	    \rarr Hom(m,X) = X^{m} $ is the canonical map $ (x_{1},...,x_{n}) \mapsto (x_{\phi(1)},...,x_{\phi(n)}) $ [Dur 4.1.4] (1.2.2.1).
        According to 1.2.1 a description of algbraic monad can be obtain as sequence of sets $ \{\Sigma(n)\}_{n \geqslant 0} $
        and maps $ \Sigma(\phi) : \Sigma(\textbf{n}) \rarr \Sigma(\textbf{m}) $, $ \phi : \textbf{m} \rarr \textbf{n} $. Now we can combine this discription and (1.2.2.1). Hence we obtain a colletion of multiplicaiton maps $ \mu_{n}^{k} : \Sigma(k) \times
        \Sigma(n)^{k} \rarr \Sigma(n) $ subject to (1.2.2.1). There is a convinient nonations for this namely 
        $ t(x_{1},...,x_{k}) \equiv \mu_{n}^{(k)}(t; x_{1},...,x_{k}) $, where $t$ is siad an operation the arity of $k$. 
        We also obtain an \textit{identity} \textbf{e} equal
        $ \epsilon_{1}(1) \in \Sigma(1) $ by Yoneda lemma.
        \subsubsection{Generalized rings.} There is a notion of \textit{commutativity} for algebraic monad
        ([Dur 5.1.1] for more details). The \textit{generalized ring} is commutative monad. The category of generalized rings is
        full subcategory of $ Monads_{alg}(Sets) $. We denote this category by $ GenR $.
	
	%\subsection{Modules over an algebraic monads.}
	

\section{The homotopy framework.}
The main goal of homotopy theory is to study of objects of certain category up to "weak equvalence", i.e. the \textit{localization} process. The modern approch is to consider of the homotopy categories and the derived functors.
	\subsection{Homotopy categories.}
	A \textit{homotopical category} is category $\mc{M}$ with a class $\mc{W}$ of morphism called \textit{weak equivalence} that contains all the identities and subject to \textit{2-of-6 property} such that if $ hg $ and $ gf $ are in $\mc{W}$ so are 
	$ f, g, h, hgf $.
	\newline
	\newline
	2-of-6 property is stronger then common 2-of-3 property in definition of the model category. Nonetheless, the weak
	equivalences of any model category statisfy the 2-of-6 property. The \textit{minimal category} is the simple example of the homotopical category in which weak equivalence is taken to be isomorphisms. We can consider for any homotopical category $\mc{M}$
	a \textit{homotopy category} Ho$ (\mc{M}) $, obtained by formal inverting the weak equivalences. Thus we get a localization
	functor $ \gamma : \mc{M} \rarr $ Ho$ (\mc{M}) $ which is universal among functors that invert the weak equivalences. In genral,
	there is some theoretical issues because Ho$(\mc{M})$ need not have samll hom-sets. Hence there is methods that are avalible 
	to enusre local smallness. For example, a categories admitting a Quillen model structure.
	
	\subsection{Derived functors.} 
		\subsubsection{Homopical functors.}
		Let $\mc{M}$, $\mc{N}$ be a homotopical actegories, and Ho$(\mc{M})$, Ho$(\mc{N})$ be their homotopy categories, with localization functors $\gamma : \mc{M} \rarr$ Ho$(\mc{M})$, $\delta : \mc{N} \rarr$ Ho$(\mc{N})$. The functor is said \textit{homotopical} if it preserves weak equivalences. If $F$ is homotopical, then by universal property $\delta F $ 
		induces a unique functor $ \tilde{F} $ commuting with localizations. %making the following diagrm commute.
%		\begin{center} 
%			\begin{tikzcd}
%				A \arrow[r, "\phi"] \arrow[d]
%				& B \arrow[d, "\psi"] \\
%					C \arrow[r, "\eta" ]
%				& D
%			\end{tikzcd}
%		\end{center}
		\subsubsection{Derived functors.}
		In general, for non-homotopical functor, there is a notion of \textit{derived functor} that is the closest homotopical aproximation. We are going to define a several notioins related with derived functor (all taken from [Shu]). 
		\newline
		\newline
		1) A \textit{total left ferived funtor} of $F : \mc{M} \rarr \mc{N} $ as  left Kan extension of $\delta F$ along $\gamma$ 
		and denoted \textbf{L}$F$. 
		\newline
		2) A \textit{left derived functor} of $F$ is a functor $\textbf{L} F$ : $\mc{M} \rarr $ Ho($\mc{N}$)
		equipped with comprassion map $\textbf{L} F$ $\rarr$ $\delta F$ such that $\textbf{L} F$ is homotopical and terminal among
		homotopical functors equiped with maps $\delta F$.
		\newline
		3) A \textit{point-set-left} derived functor is a functor $ \mathds{L} F : \mc{M} \rarr \mc{N} $ equipped with comprasion
		map $ \mathds{L} F \rarr F $ such that the induced map $ \delta \mathds{L} F \rarr \delta F $ makes $ \delta \mathds{L} F $
		into a left derived functor of $F$. 
%		\begin{center} 
%			\begin{tikzcd}
%				A \arrow[r, "\phi"] \arrow[d]
%				& B \arrow[d, "\psi"] \\
%					C \arrow[r, "\eta" ]
%				& D
%			\end{tikzcd}
%		\end{center}
       \subsubsection{Deformations.}
       In this section we describe derived functors via deformations.
       \newline
       \newline
       A \textit{left defromations} is functor $ Q : \mc{M} \rarr \mc{M} $ togther with natural weak equivalence 
       $ q : Q \tilde{\rarr} \oid_{\mc{M}} $. It is easy to see that Q is homotopical by 2-of-3 property. $\mc{M}_{Q}$ is
       called a \textit{left deforamtion retract} [Shu] or category of \textit{cofibrant objects} [Ri].
       By universal property, there are functors Ho($\mc{M}$) $\rarr$ Ho($\mc{M}_{Q}$) and Ho($\mc{M}_{Q}$) $\rarr$ Ho($\mc{M}$).
       Hence there is an equivalence of categories Ho($\mc{M}_{Q}$) $\cong$ Ho($\mc{M}$).
       \newline
       \newline
       (2.2.3.1) \textbf{Lemma}([Ri 2.2.8]). If $ F : \mc{M} \rarr \mc{N} $ has a left deformarion $ q : Q \tilde{\rarr} \oid $, then
       $ F Q $ is a left derived functor of $F$.
       \subsubsection{Example.} The most important example for us is the calssical derived functor between abelina categories from
       homological algebra.
       \newline
       \newline
       Let $\mc{A}$ be any abelian category with sufficiently many projective objects e.g. $ \mc{A} = Mod_{R}$ category of left $R$-modules for classical ring $R$. Le $ Ch_{\geqslant 0}(R) $ be a category bounded below chain complexes and quasi-isomorphisms is taken as weak equivalences. For any $R$-module $X$ there exists a projective module $P$ and 
       surjection $ P \twoheadrightarrow X $. We can define a \textit{projective resolution}, a chain complex $J_{\bullet} \in Ch_{\geqslant 0}$ equipped with a quasi-isomorphism $p : J_{\bullet} \tilde{\rarr} X$. The operation of taken of a projective reoslutions $ Q : Ch_{\geqslant 0}(R) \rarr Ch_{\geqslant 0}(R) $ defines a left derformaion $ q : Q \tilde{\rarr} \oid $. Any additive functors $ F : Mod_{R} \rarr Mod_{S} $ induces a functor
       $ F_{\bullet} : Ch_{\geqslant 0}(R) \rarr Ch_{\geqslant 0}(S) $ that preserves chain homotopy equivalences. Because
       any quasi-isomorphism between non-negatively chain complexes of projetive objects is a chian homotopy equivalence, $ F $ has a
       left derived functor
       $ Mod_{R} \xrarr[]{deg0} Ch_{\geqslant 0}(R) \xrarr[]{\mathds{L} F} Ch_{\geqslant 0}(S) \xrarr[]{H_{0}} Mod_{S} $.     	
	
	\subsection{Model categories.}
	 A model categories is commonly used homotopy framework. The aspects of theory of model atefories widely represented in
     many litratures e.g. [Hir], [Hov]. We make a brief description.
	 \subsubsection{Model Structure.}
	 A \textit{model structure} on a category $\mc{C}$ is three distinguished classes of morphisms of $\mc{C}$ 
	 called weak equivalences, cofibrations and fibrations subject to the following properties 1)..4).
	 \newline
	 \newline
	 Let's introduce some definitoins before. A map $ f : A \rarr B $ is said a \textit{retract} of $ g : C \rarr D$ in categoty $\mc{C}$ if $f$ is retract of $g$ in category of maps Map$(\mc{C})$ of $\mc{C}$, i.e. if following diagram is commutative.
	 
	 		\begin{center} 
	 			\begin{tikzcd}
	 				A \arrow[d, "f"] \arrow[r] & C \arrow[d, "g"] \arrow[r] & A \arrow[d, "f"] \\
	 				B \arrow[r] & D \arrow[r] & B
	 			\end{tikzcd}
	 		\end{center}
 	 
 	 Let $i : A \rarr B$ and $ g : X \rarr Y $ be a morphisms in $\mc{C}$. Then $i$ is said has a left lifting property respect (LLP) to $g$, or $g$ has right lifitn propety (RLP) respect to $i$, if $ u : A \rarr X $ and $ v : B \rarr Y $, such that $ v i = f u $,
 	 there exists $ h : B \rarr X $, such that $ h i = u $ and $ g h = v $, i.e. followong diagram is comutative:
 	 
 	 		\begin{center} 
 	 			\begin{tikzcd}
 	 				A \arrow[r, "u"] \arrow[d, "i"] & X \arrow[d, "g"] \\
 	 				B \arrow[r, "v"] \arrow[ur, dashrightarrow, "h"] & Y
 	 			\end{tikzcd}
 	 		\end{center}
  			
  	 A map is said a \textit{trivial (co)fibration} if it is both (co)fibration and trivial equivalence.
  	 \newline
  	 \newline	
	 1)(2-of-3) If $ f, g \in$ Mor($\mc{C}$) and $gf \in$ Mor($\mc{C}$) and two of $f, g, gf$ are weak equivalences, then is the third.
	 \newline
	 \newline
	 2)(Retracts) Closeness with respect to retracts for each of three classes.
	 \newline
	 \newline
	 3)(Lifiting) Any cofibrations have the LLP with respect to all trivial fibrations, and any fibrations have the RLP with respect to all trivial cofibration.
	 \newline
	 \newline
	 4)(Factorization) For any morphism there exists both of factorizations into trivial cofibration followed by fibration, or
	 into a cofibration followed by trivial fibration.
	 \newline
	 \newline
	 A \textit{model category} $\mc{C}$ is a complete and cocompete category equiped with a model structure.
	 
	 \subsubsection{Homtopies.}
	 Let $\mc{C}$ be a model category. Then $\Ho(\mc{C})$ is corresponding homotopy category and
	 $ \gamma : \mc{C} \rarr \Ho(\mc{C}) $ is a localization functor. We are going to define a homotopy equivalence relation on $\Hom_{\Ho(\mc{C})}(\gamma X, \gamma Y) = [X, Y]$.
	 \newline
	 \newline
	 Let's fix maps $f,g : A \rarr B $. A \textit{cylinder object} $ A \times I $ for object $ A $ is a morphism 
	 $ \nabla_{A} : A \sqcup A \xrarr[]{(i_0, i_1)} A \times I \xrarr[]{\sigma} A $, such that $(i_{1}, i_{2})$ is a cofibration and $\sigma$ is a weak equivalence. A path object $ B^I $ for object $B$ is a morphism 
	 $ \Delta_{B} : B \xrarr[]{s} B^I \xrarr[]{(j_{0}, j_{1})} B \times B $, such that $s$ is a weak equivalence and
	 $ (j_{0}, j_{1}) $ is a fibration.
	 \newline
	 \newline 
	 A maps $f$ and $g$ are said \textit{left} (resp. \textit{right}) \textit{homotopic}, written $ f \overset l\sim g $
	 (resp. $ f \overset r\sim g $)  if there exists \textit{left} (resp. \textit{right homotopy}) from $f$ to $g$, i.e. there exists a map $ H : A' \rarr B $ (resp. $ K : A \rarr B' $) for some cylinder (resp. path) object $A'$ (resp. $B'$) such that $ H i_{0} = f $ and $ H i_{1} = g $ (resp. $ j_{0} K = f $ and $ j_{1} K = g $). A maps are said \textit{homoptopic}, written $ f \sim g $, if they are both left and right homotopic. We write $ \pi_{l} $ (resp. $\pi_{r}$) for the qotient of $\Hom(A,B)$ 
	 with respect to the equivalence relation generated by $\overset l\sim$(resp. $\overset r\sim$). If $A$ is cofibrant and
	 $B$ is fibrant, then $\overset l\sim$ and $\overset r\sim$ coinside on $\Hom(A,B)$ and denoted by $\sim$. We write
	 $\pi(A,B)$ for $\Hom(A,B) / \sim$.
	 \newline
	 \newline
	 Let $\mc{C}_{c}$, $\mc{C}_{f}$ and $\mc{C}_{cf}$ be a full subcotegories of category $\mc{C}$ sonsisiting of all cofibrant,
	 fibrant and fibrant-cofibrant objects respetvely and $\Ho(\mc{C}_{c})$, $\Ho(\mc{C}_{f})$ and $\Ho(\mc{C}_{cf})$ be corresponding homotopy categories. Denote by $\pi \mc{C}_{c}$ the category with the same object as $\mc{C}$ and morphisms given by
	 $\Hom_{\pi \mc{C}_{c}}(A, B) = \pi^{r}(A, B)$ and define similary for $\mc{C}_{f}$ and $\mc{C}_{cf}$. Then funcotrs
	 $ \Ho(\mc{C}_{c}) \rarr \Ho(\mc{C}), \newline \Ho(\mc{C}_{f}) \rarr \Ho(\mc{C}), \pi \mc{C}_{f} \rarr \Ho(\mc{C}) $ 
	 are equvalences of catefories [Quillen]. 
	 \newline
	 \newline
	 \subsubsection{Higher homotopy groups.} Given model category $\mc{C}$, we define notions of the suspension and the loop.
	 A \textit{suspension object} $\Sigma A$ of a cofibrant odject is a pushout with respect to the map $ A \sqcup A \rarr 0 $ of the map $ A \sqcup A \rarr A \times I $. Similary, a \textit{loop object} $\Omega B$ is pullback of  $ 0 \rarr B \times B $ by 
	 $ (j_{0}, j_{1}) : B^{I} \rarr B \times B $. The objects $\Sigma A$ and $\Omega B$, called also cofiber and fiber respectively, are a particular examples of the homotopy limit and the homotopy colimit in modern homotopy theory.
	 \newline
	 \newline
	 Given any two morphisms $ f, g : A \rarr B  $ with cofibrant $A$ and fibrant $B$, we can define a notion of the left homotopy betwee leftn homotopies. This notions is a equivalence relation and define set $ \pi_{1}^{l} (A,B;f,g) $ of homotopy classes of
	 homotopies $ h : f \sim g $. Simmilary, there exists a dual constraction $ \pi_{1}^{r} (A,B;f,g)$ that turns out to be isomorphic.
	 Thus we  denote this by $ \pi_{1} (A,B;f,g) $. Let suppose that category $\mc{C}$ has a null object $0_{\mc{C}}$. Then, we put
	 $ \pi_{1}(A,B) \equiv \pi_{1}^{l} (A,B;0,0) $ where $0$ is a zero map. This is a group. Thus we get a functor
	 $ A, B \longmapsto [A, B]_{1} $, $ (\Ho (\mc{C}))^{o}  \times  \Ho (\mc{C}) \rarr Grps $, whenever $A$ is a cofibrantand and
	 $B$ is a fibrant.
	 \newline
	 \newline   
	 \textbf{Theorem}([Quillen]). There are two functors $ \Ho(\mc{C}) \rarr \Ho({\mc{C}}) $, called the \textit{suspension} and the
	 \textit{loop} functor, such that $ [\Sigma A, B] \cong [A, B]_{1} \cong [A, \Omega B] $.
	  
%\section{Cubical Sets}
%The category of cubical sets is an alternative for the category of simplicial sets being is one of the main tool of homotopy theory.
%
%\label{sec:org75a4632}
%	\subsection{The category of cubical sets.}
%	\subsection{Monoidal closed structure.}
%	\subsection{Homotopy propirties of cubical sets.}
%
%\section{Homotopical algerbra of cubical sets.}
%
%\section{Cubical $\Sigma$-modules.}

\newpage


\begin{thebibliography}{9}
	
	\bibitem{article}
	Nikolai Durov. New Approach to Arakelov Geometry.
	
	\bibitem{article}
	Emily Riehl. Categorical homotopy theory.
	
	\bibitem{article}
	Michael Shulman. Homotopy limits and colimits and enriched homotopy theory.
	
	\bibitem{article}
	Mark Hovey. Model categories.
	
\end{thebibliography}



\end{document}




